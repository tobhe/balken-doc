\documentclass[runningheads,a4paper]{llncs}

\usepackage{amssymb}
\setcounter{tocdepth}{3}
\usepackage{graphicx}
\usepackage{subfig}
%\linespread{2}

\usepackage{url}
\usepackage{csquotes}
\newcommand{\keywords}[1]{\par\addvspace\baselineskip
\noindent\keywordname\enspace\ignorespaces#1}

\usepackage{listings}
\usepackage{color}
\usepackage{enumitem}
\usepackage{hyperref}

\definecolor{dkgreen}{rgb}{0,0.6,0}
\definecolor{gray}{rgb}{0.5,0.5,0.5}
\definecolor{mauve}{rgb}{0.58,0,0.82}

\lstset{frame=tb,
  language=C++,
  aboveskip=3mm,
  belowskip=3mm,
  showstringspaces=false,
  columns=flexible,
  basicstyle={\small\ttfamily},
  numbers=left,
  numberstyle=\tiny\color{gray},
  keywordstyle=\color{blue},
  morekeywords={vector},
  commentstyle=\color{dkgreen},
  stringstyle=\color{mauve},
  breaklines=true,
  breakatwhitespace=true,
  tabsize=3
}

\begin{document}

\mainmatter  % start of an individual contribution

% first the title is needed
\title{balken: a modern C++ barcode detection library using mathematical morphology}

% a short form should be given in case it is too long for the running head
\titlerunning{balken: modern C++ barcode detection}

%
\author{Tobias Heider}
%
\authorrunning{Tobias Heider}
% (feature abused for this document to repeat the title also on left hand pages)

\institute{Practical Course "Advanced Software Development with Modern C++"\\Summer Term 2018\\Institute for Computer Science\\
Ludwig Maximilians University of Munich\\
}

\maketitle


\section{Introduction}

\section{Methods of barcode detection}
Various methods of barcode detection have been proposed. This section will
present three alorithms using different approaches to find a barcodes location
in an image.
All presented methods work on greyscale images as they allow
equally effective detection of black and white barcodes are require less
computing power. Most methods use similar means of preprocessing on the input images
before applying the actual detection algorithms. Often a gaussian kernel filter
is applied to reduce image noice. Histogram equalization can be used to increase
the input images contrast by spreading out the pixel intensities over the full spectrum.

\subsection{Image Scanning}

In~\cite{tekin2009algorithm} Telkin and Coughlan present a method to detect one
dimensional UPC-A barcodes.
The algorithm first scans the image's gradient in 4 different
directions. If a bar's edge is detected the image is scanned again in the
perpendicular direction to detect the bar's second edge. Regions with
consecutive segments that have similar beginnings and ends are saved as barcode
candidates. Candidates are then filtered by the calculated entropy value of
their gradients angles, as all bars are expected to have the same angle.

\subsection{Machine Learning}

The procedure proposed by Hansen et al. in~\cite{hansen2017real} can detect both
1D barcodes and QR codes using the deep learning-based YOLO detector from
~\cite{redmon2017yolo9000}. The network is trained with 1D and 2D barcodes in a
fixed input format of 416x416. A regression network is used to find the amount
of rotation in training barcodes that leads to the fastest detection rates.
Drawbacks of this method are the fixed input size as well as the requirement to
train a new network for different barcode types.

\subsection{Morphology}

\bibliography{literature}
\bibliographystyle{plain}

\end{document}
